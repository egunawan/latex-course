\documentclass{beamer}

%
% Common preamble for all three parts.
%

\usepackage[english]{babel}
\usepackage{amsmath}
\usepackage{color}
\usepackage{minted}
\usepackage{hyperref}
\usepackage{multicol}
\usepackage{tabularx}
\usepackage{tikz}

% only inline todonotes work
\usepackage{xkeyval}
\usepackage[textsize=small]{todonotes}
\presetkeys{todonotes}{inline}{}

\usetikzlibrary{shapes,arrows,positioning,shadows}

% no nav buttons
\usenavigationsymbolstemplate{}

\newcommand{\bftt}[1]{\textbf{\texttt{#1}}}
\newcommand{\comment}[1]{{\color[HTML]{008080}\textit{\textbf{\texttt{#1}}}}}
\newcommand{\cmd}[1]{{\color[HTML]{008000}\bftt{#1}}}
\newcommand{\bs}{\char`\\}
\newcommand{\cmdbs}[1]{\cmd{\bs#1}}
\newcommand{\lcb}{\char '173}
\newcommand{\rcb}{\char '175}
\newcommand{\cmdbegin}[1]{\cmdbs{begin\lcb}\bftt{#1}\cmd{\rcb}}
\newcommand{\cmdend}[1]{\cmdbs{end\lcb}\bftt{#1}\cmd{\rcb}}

\newcommand{\wllogo}{\textbf{Overleaf}}

% this is where the example source files are loaded from
% do not include a trailing slash
\newcommand{\fileuri}{https://raw.github.com/egunawan/latex-course/master/en}

\newcommand{\wlserver}{https://www.overleaf.com}
\newcommand{\wlnewdoc}[1]{\wlserver/docs?snip\_uri=\fileuri/#1\&splash=none}

\def\tikzname{Ti\emph{k}Z}

% from http://tex.stackexchange.com/questions/5226/keyboard-font-for-latex
\newcommand*\keystroke[1]{%
  \tikz[baseline=(key.base)]
    \node[%
      draw,
      fill=white,
      drop shadow={shadow xshift=0.25ex,shadow yshift=-0.25ex,fill=black,opacity=0.75},
      rectangle,
      rounded corners=2pt,
      inner sep=1pt,
      line width=0.5pt,
      font=\scriptsize\sffamily
    ](key) {#1\strut}
  ;
}
\newcommand{\keystrokebftt}[1]{\keystroke{\bftt{#1}}}

% stolen from minted.dtx
\newenvironment{exampletwoup}
  {\VerbatimEnvironment
   \begin{VerbatimOut}{example.out}}
  {\end{VerbatimOut}
   \setlength{\parindent}{0pt}
   \fbox{\begin{tabular}{l|l}
   \begin{minipage}{0.55\linewidth}
     \inputminted[fontsize=\small,resetmargins]{latex}{example.out}
   \end{minipage} &
   \begin{minipage}{0.35\linewidth}
     \input{example.out}
   \end{minipage}
   \end{tabular}}}

\newenvironment{exampletwouptiny}
  {\VerbatimEnvironment
   \begin{VerbatimOut}{example.out}}
  {\end{VerbatimOut}
   \setlength{\parindent}{0pt}
   \fbox{\begin{tabular}{l|l}
   \begin{minipage}{0.55\linewidth}
     \inputminted[fontsize=\scriptsize,resetmargins]{latex}{example.out}
   \end{minipage} &
   \begin{minipage}{0.35\linewidth}
     \setlength{\parskip}{6pt plus 1pt minus 1pt}%
     \raggedright\scriptsize\input{example.out}
   \end{minipage}
   \end{tabular}}}

\newenvironment{exampletwouptinynoframe}
  {\VerbatimEnvironment
   \begin{VerbatimOut}{example.out}}
  {\end{VerbatimOut}
   \setlength{\parindent}{0pt}
   \begin{tabular}{l|l}
   \begin{minipage}{0.55\linewidth}
     \inputminted[fontsize=\scriptsize,resetmargins]{latex}{example.out}
   \end{minipage} &
   \begin{minipage}{0.35\linewidth}
     \setlength{\parskip}{6pt plus 1pt minus 1pt}%
     \raggedright\scriptsize\input{example.out}
   \end{minipage}
   \end{tabular}}

\title{Math 2794W An Interactive Introduction to \LaTeX}
\author{Original slides written by Dr John D. Lees-Miller (modified with permission)}
\titlegraphic{%
\includegraphics[height=36pt]{overleaf}\\[1em]
\includegraphics[height=24pt]{UoB-logo}
}


\subtitle{Part 2: Structured Documents \& More}

\begin{document}

%%%%%%%%%%%%%%%%%%%%%%%%%%%%%%%%%%%%%%%%%%%%%%%%%%%%%%%%%%%%%%%%%%%%%%%%%%%%%%%
%%%%%%%%%%%%%%%%%%%%%%%%%%%%%%%%%%%%%%%%%%%%%%%%%%%%%%%%%%%%%%%%%%%%%%%%%%%%%%%
%%%%%%%%%%%%%%%%%%%%%%%%%%%%%%%%%%%%%%%%%%%%%%%%%%%%%%%%%%%%%%%%%%%%%%%%%%%%%%%
\begin{frame}
\titlepage
\end{frame}

%%%%%%%%%%%%%%%%%%%%%%%%%%%%%%%%%%%%%%%%%%%%%%%%%%%%%%%%%%%%%%%%%%%%%%%%%%%%%%%
%%%%%%%%%%%%%%%%%%%%%%%%%%%%%%%%%%%%%%%%%%%%%%%%%%%%%%%%%%%%%%%%%%%%%%%%%%%%%%%
%%%%%%%%%%%%%%%%%%%%%%%%%%%%%%%%%%%%%%%%%%%%%%%%%%%%%%%%%%%%%%%%%%%%%%%%%%%%%%%
\section{Structured Documents}

%%%%%%%%%%%%%%%%%%%%%%%%%%%%%%%%%%%%%%%%%%%%%%%%%%%%%%%%%%%%%%%%%%%%%%%%%%%%%%%
%%%%%%%%%%%%%%%%%%%%%%%%%%%%%%%%%%%%%%%%%%%%%%%%%%%%%%%%%%%%%%%%%%%%%%%%%%%%%%%
%%%%%%%%%%%%%%%%%%%%%%%%%%%%%%%%%%%%%%%%%%%%%%%%%%%%%%%%%%%%%%%%%%%%%%%%%%%%%%%
\begin{frame}{Outline}
\begin{multicols}{2}
\tableofcontents[currentsection]
\end{multicols}
\end{frame}

%%%%%%%%%%%%%%%%%%%%%%%%%%%%%%%%%%%%%%%%%%%%%%%%%%%%%%%%%%%%%%%%%%%%%%%%%%%%%%%
%%%%%%%%%%%%%%%%%%%%%%%%%%%%%%%%%%%%%%%%%%%%%%%%%%%%%%%%%%%%%%%%%%%%%%%%%%%%%%%
%%%%%%%%%%%%%%%%%%%%%%%%%%%%%%%%%%%%%%%%%%%%%%%%%%%%%%%%%%%%%%%%%%%%%%%%%%%%%%%
\begin{frame}{\insertsection}
\begin{itemize}
\item In Part 1, we learned about commands and environments for typesetting text
and mathematics.
\item Now, we'll learn about commands and environments for structuring
documents.
\item You can try out the new commands in Overleaf:
\end{itemize}
\vskip 2em
\begin{center}
\fbox{\href{\wlnewdoc{basics.tex}}{%
Click here to open the example document in \wllogo{}}}
\\[1ex]\scriptsize{}
For best results, please use \href{http://www.google.com/chrome}{Google Chrome} or a recent \href{http://www.mozilla.org/en-US/firefox/new/}{FireFox}.
\end{center}
\vskip 2ex
\begin{itemize}
\item Let's get started!
\end{itemize}
\end{frame}

%%%%%%%%%%%%%%%%%%%%%%%%%%%%%%%%%%%%%%%%%%%%%%%%%%%%%%%%%%%%%%%%%%%%%%%%%%%%%%%
%%%%%%%%%%%%%%%%%%%%%%%%%%%%%%%%%%%%%%%%%%%%%%%%%%%%%%%%%%%%%%%%%%%%%%%%%%%%%%%
%%%%%%%%%%%%%%%%%%%%%%%%%%%%%%%%%%%%%%%%%%%%%%%%%%%%%%%%%%%%%%%%%%%%%%%%%%%%%%%
\subsection{Title and Abstract}
\begin{frame}[fragile]{\insertsubsection}
\begin{itemize}{\small
\item Tell \LaTeX{} the \cmdbs{title} and \cmdbs{author} names in the preamble.
\item Then use \cmdbs{maketitle} in the document to actually create the title.
\item Use the \bftt{abstract} environment to make an abstract.
}\end{itemize}
\begin{minipage}{0.55\linewidth}
\inputminted[fontsize=\scriptsize,frame=single,resetmargins]{latex}%
  {structure-title.tex}
\end{minipage}
\begin{minipage}{0.35\linewidth}
\includegraphics[width=\textwidth,clip,trim=2.2in 7in 2.2in 2in]{structure-title.pdf}
\end{minipage}
\end{frame}

%%%%%%%%%%%%%%%%%%%%%%%%%%%%%%%%%%%%%%%%%%%%%%%%%%%%%%%%%%%%%%%%%%%%%%%%%%%%%%%
%%%%%%%%%%%%%%%%%%%%%%%%%%%%%%%%%%%%%%%%%%%%%%%%%%%%%%%%%%%%%%%%%%%%%%%%%%%%%%%
%%%%%%%%%%%%%%%%%%%%%%%%%%%%%%%%%%%%%%%%%%%%%%%%%%%%%%%%%%%%%%%%%%%%%%%%%%%%%%%
\subsection{Sections}
\begin{frame}{\insertsubsection}
\begin{itemize}{\small
\item Just use \cmdbs{section} and \cmdbs{subsection}.
\item Can you guess what \cmdbs{section*} and \cmdbs{subsection*} do?
}\end{itemize}
\begin{minipage}{0.55\linewidth}
\inputminted[fontsize=\scriptsize,frame=single,resetmargins]{latex}%
  {structure-sections.tex}
\end{minipage}
\begin{minipage}{0.35\linewidth}
\includegraphics[width=\textwidth,clip,trim=1.5in 6in 4in 1in]{structure-sections.pdf}
\end{minipage}
\end{frame}

%%%%%%%%%%%%%%%%%%%%%%%%%%%%%%%%%%%%%%%%%%%%%%%%%%%%%%%%%%%%%%%%%%%%%%%%%%%%%%%
%%%%%%%%%%%%%%%%%%%%%%%%%%%%%%%%%%%%%%%%%%%%%%%%%%%%%%%%%%%%%%%%%%%%%%%%%%%%%%%
%%%%%%%%%%%%%%%%%%%%%%%%%%%%%%%%%%%%%%%%%%%%%%%%%%%%%%%%%%%%%%%%%%%%%%%%%%%%%%%
\subsection{Labels and Cross-References}
\begin{frame}[fragile]{\insertsubsection}
\begin{itemize}{\small
\item Use \cmdbs{label} and \cmdbs{ref} for automatic numbering.
\item The \bftt{amsmath} package provides \cmdbs{eqref} for referencing
equations.
}\end{itemize}
\begin{minipage}{0.55\linewidth}
\inputminted[fontsize=\scriptsize,frame=single,resetmargins]{latex}%
  {structure-crossref.tex}
\end{minipage}
\begin{minipage}{0.35\linewidth}
\includegraphics[width=\textwidth,clip,trim=1.8in 6in 1.6in 1in]{structure-crossref.pdf}
\end{minipage}
\end{frame}

%%%%%%%%%%%%%%%%%%%%%%%%%%%%%%%%%%%%%%%%%%%%%%%%%%%%%%%%%%%%%%%%%%%%%%%%%%%%%%%
%%%%%%%%%%%%%%%%%%%%%%%%%%%%%%%%%%%%%%%%%%%%%%%%%%%%%%%%%%%%%%%%%%%%%%%%%%%%%%%
%%%%%%%%%%%%%%%%%%%%%%%%%%%%%%%%%%%%%%%%%%%%%%%%%%%%%%%%%%%%%%%%%%%%%%%%%%%%%%%
\subsection{Exercise}
\begin{frame}[fragile]{Structured Documents Exercise}

\begin{block}{Typeset this short paper in \LaTeX:
\footnote{From \url{http://pdos.csail.mit.edu/scigen/}, a random
paper generator.}}
\begin{center}
\fbox{\href{\fileuri/structure-exercise-solution.pdf}{%
Click to open the paper}}
\end{center}
Make your paper look like this one. Use \cmdbs{ref} and \cmdbs{eqref} to avoid
explicitly writing section and equation numbers into the text.
\end{block}
\vskip 2ex
\begin{center}
\fbox{\href{\wlnewdoc{structure-exercise.tex}}{%
Click to open this exercise in \wllogo{}}}
\end{center}

\begin{itemize}
\item Once you've tried,
\fbox{\href{\wlnewdoc{structure-exercise-solution.tex}}{%
click here to see my solution}}.
\end{itemize}
\end{frame}

%%%%%%%%%%%%%%%%%%%%%%%%%%%%%%%%%%%%%%%%%%%%%%%%%%%%%%%%%%%%%%%%%%%%%%%%%%%%%%%
%%%%%%%%%%%%%%%%%%%%%%%%%%%%%%%%%%%%%%%%%%%%%%%%%%%%%%%%%%%%%%%%%%%%%%%%%%%%%%%
%%%%%%%%%%%%%%%%%%%%%%%%%%%%%%%%%%%%%%%%%%%%%%%%%%%%%%%%%%%%%%%%%%%%%%%%%%%%%%%
\section{Figures and Tables}

%%%%%%%%%%%%%%%%%%%%%%%%%%%%%%%%%%%%%%%%%%%%%%%%%%%%%%%%%%%%%%%%%%%%%%%%%%%%%%%
%%%%%%%%%%%%%%%%%%%%%%%%%%%%%%%%%%%%%%%%%%%%%%%%%%%%%%%%%%%%%%%%%%%%%%%%%%%%%%%
%%%%%%%%%%%%%%%%%%%%%%%%%%%%%%%%%%%%%%%%%%%%%%%%%%%%%%%%%%%%%%%%%%%%%%%%%%%%%%%
\begin{frame}{Outline}
\begin{multicols}{2}
\tableofcontents[currentsection]
\end{multicols}
\end{frame}

%%%%%%%%%%%%%%%%%%%%%%%%%%%%%%%%%%%%%%%%%%%%%%%%%%%%%%%%%%%%%%%%%%%%%%%%%%%%%%%
%%%%%%%%%%%%%%%%%%%%%%%%%%%%%%%%%%%%%%%%%%%%%%%%%%%%%%%%%%%%%%%%%%%%%%%%%%%%%%%
%%%%%%%%%%%%%%%%%%%%%%%%%%%%%%%%%%%%%%%%%%%%%%%%%%%%%%%%%%%%%%%%%%%%%%%%%%%%%%%
\subsection{Graphics}
\begin{frame}[fragile]{\insertsubsection}
\begin{itemize}
\item Requires the \bftt{graphicx} package, which provides the
\cmdbs{includegraphics} command.
\item Supported graphics formats include JPEG, PNG and PDF (usually).
\end{itemize}
\begin{exampletwouptiny}
\includegraphics[
  width=0.5\textwidth]{gerbil}

\includegraphics[
  width=0.3\textwidth,
  angle=270]{gerbil}
\end{exampletwouptiny}

\tiny{Image license: \href{https://pixabay.com/en/animal-apple-attractive-beautiful-1239390/}{CC0}}
\end{frame}

%%%%%%%%%%%%%%%%%%%%%%%%%%%%%%%%%%%%%%%%%%%%%%%%%%%%%%%%%%%%%%%%%%%%%%%%%%%%%%%
%%%%%%%%%%%%%%%%%%%%%%%%%%%%%%%%%%%%%%%%%%%%%%%%%%%%%%%%%%%%%%%%%%%%%%%%%%%%%%%
%%%%%%%%%%%%%%%%%%%%%%%%%%%%%%%%%%%%%%%%%%%%%%%%%%%%%%%%%%%%%%%%%%%%%%%%%%%%%%%
\begin{frame}[fragile]{Interlude: Optional Arguments}
\begin{itemize}
\item We use square brackets \keystrokebftt{[} \keystrokebftt{]} for optional
arguments, instead of braces \keystrokebftt{\{} \keystrokebftt{\}}.
\item \cmdbs{includegraphics} accepts optional arguments that allow you to transform the
image when it is included. For example, \bftt{width=0.3\cmdbs{textwidth}} makes
the image take up 30\% of the width of the surrounding text (\cmdbs{textwidth}).
\item \cmdbs{documentclass} accepts optional arguments, too. Example:
\mint{latex}|\documentclass[12pt,twocolumn]{article}|
\vskip 3ex
makes the text bigger (12pt) and puts it into two columns.
\item Where do you find out about these? See the slides at the end of this
presentation for links to more information.
\end{itemize}
\end{frame}

%%%%%%%%%%%%%%%%%%%%%%%%%%%%%%%%%%%%%%%%%%%%%%%%%%%%%%%%%%%%%%%%%%%%%%%%%%%%%%%
%%%%%%%%%%%%%%%%%%%%%%%%%%%%%%%%%%%%%%%%%%%%%%%%%%%%%%%%%%%%%%%%%%%%%%%%%%%%%%%
%%%%%%%%%%%%%%%%%%%%%%%%%%%%%%%%%%%%%%%%%%%%%%%%%%%%%%%%%%%%%%%%%%%%%%%%%%%%%%%
\subsection[fragile]{Floats}
\begin{frame}{\insertsubsection}
\begin{itemize}
\item Allow \LaTeX{} to decide where the figure will go (it can ``float'').
\item You can also give the figure a caption, which can be referenced with
\cmdbs{ref}.
\end{itemize}
\begin{minipage}{0.55\linewidth}
\inputminted[fontsize=\scriptsize,frame=single,resetmargins]{latex}%
  {media-graphics.tex}
\end{minipage}
\begin{minipage}{0.35\linewidth}
\includegraphics[width=\textwidth,clip,trim=2in 5in 3in 1in]{media-graphics.pdf}
\end{minipage}

\tiny{Image license: \href{https://pixabay.com/en/animal-apple-attractive-beautiful-1239390/}{CC0}}
\end{frame}

%%%%%%%%%%%%%%%%%%%%%%%%%%%%%%%%%%%%%%%%%%%%%%%%%%%%%%%%%%%%%%%%%%%%%%%%%%%%%%%
%%%%%%%%%%%%%%%%%%%%%%%%%%%%%%%%%%%%%%%%%%%%%%%%%%%%%%%%%%%%%%%%%%%%%%%%%%%%%%%
%%%%%%%%%%%%%%%%%%%%%%%%%%%%%%%%%%%%%%%%%%%%%%%%%%%%%%%%%%%%%%%%%%%%%%%%%%%%%%%
\subsection{Tables}
\begin{frame}[fragile]{\insertsubsection}
\begin{itemize}
\item Tables in \LaTeX{} take some getting used to.
\item Use the \bftt{tabular} environment from the \bftt{tabularx} package.
\item The argument specifies column alignment --- \textbf{l}eft, \textbf{r}ight, \textbf{r}ight.
\begin{exampletwouptiny}
\begin{tabular}{lrr}
Item   & Qty & Unit \$ \\
Widget & 1   & 199.99  \\
Gadget & 2   & 399.99  \\
Cable  & 3   & 19.99   \\
\end{tabular}
\end{exampletwouptiny}
\item It also specifies vertical lines; use \cmdbs{hline} for horizontal lines.
\begin{exampletwouptiny}
\begin{tabular}{|l|r|r|} \hline
Item   & Qty & Unit \$ \\\hline
Widget & 1   & 199.99  \\
Gadget & 2   & 399.99  \\
Cable  & 3   & 19.99   \\\hline
\end{tabular}
\end{exampletwouptiny}
\item Use an ampersand \keystrokebftt{\&} to separate columns and a double backslash \keystrokebftt{\bs}\keystrokebftt{\bs} to start a new row (like in the \bftt{align*} environment that we saw in part 1).
\end{itemize}
\end{frame}

%%%%%%%%%%%%%%%%%%%%%%%%%%%%%%%%%%%%%%%%%%%%%%%%%%%%%%%%%%%%%%%%%%%%%%%%%%%%%%%
%%%%%%%%%%%%%%%%%%%%%%%%%%%%%%%%%%%%%%%%%%%%%%%%%%%%%%%%%%%%%%%%%%%%%%%%%%%%%%%
%%%%%%%%%%%%%%%%%%%%%%%%%%%%%%%%%%%%%%%%%%%%%%%%%%%%%%%%%%%%%%%%%%%%%%%%%%%%%%%
\addtocontents{toc}{\newpage}
\section{Bibliographies}

%%%%%%%%%%%%%%%%%%%%%%%%%%%%%%%%%%%%%%%%%%%%%%%%%%%%%%%%%%%%%%%%%%%%%%%%%%%%%%%
%%%%%%%%%%%%%%%%%%%%%%%%%%%%%%%%%%%%%%%%%%%%%%%%%%%%%%%%%%%%%%%%%%%%%%%%%%%%%%%
%%%%%%%%%%%%%%%%%%%%%%%%%%%%%%%%%%%%%%%%%%%%%%%%%%%%%%%%%%%%%%%%%%%%%%%%%%%%%%%
\begin{frame}{Outline}
\begin{multicols}{2}
\tableofcontents[currentsection]
\end{multicols}
\end{frame}

%%%%%%%%%%%%%%%%%%%%%%%%%%%%%%%%%%%%%%%%%%%%%%%%%%%%%%%%%%%%%%%%%%%%%%%%%%%%%%%
%%%%%%%%%%%%%%%%%%%%%%%%%%%%%%%%%%%%%%%%%%%%%%%%%%%%%%%%%%%%%%%%%%%%%%%%%%%%%%%
%%%%%%%%%%%%%%%%%%%%%%%%%%%%%%%%%%%%%%%%%%%%%%%%%%%%%%%%%%%%%%%%%%%%%%%%%%%%%%%
\subsection{bib\TeX}
\begin{frame}[fragile]{\insertsubsection{} 1}
\begin{itemize}
\item Put your references in a \bftt{.bib} file in `bibtex' database format:
\inputminted[fontsize=\scriptsize,frame=single]{latex}{bib-example.bib}
\item Most reference managers can export to bibtex format.
\end{itemize}
\end{frame}

%%%%%%%%%%%%%%%%%%%%%%%%%%%%%%%%%%%%%%%%%%%%%%%%%%%%%%%%%%%%%%%%%%%%%%%%%%%%%%%
%%%%%%%%%%%%%%%%%%%%%%%%%%%%%%%%%%%%%%%%%%%%%%%%%%%%%%%%%%%%%%%%%%%%%%%%%%%%%%%
%%%%%%%%%%%%%%%%%%%%%%%%%%%%%%%%%%%%%%%%%%%%%%%%%%%%%%%%%%%%%%%%%%%%%%%%%%%%%%%
\begin{frame}[fragile]{\insertsubsection{} 2}
\begin{itemize}
\item Each entry in the \bftt{.bib} file has a \emph{key} that you can use to
reference it in the document. For example, \bftt{Jacobson1999Towards} is the key for this article:
\begin{minted}[fontsize=\small,frame=single]{latex}
@Article{Jacobson1999Towards,
  author = {Van Jacobson},
  ...
}
\end{minted}
\item It's a good idea to use a key based on the name, year and title.
\item \LaTeX{} can automatically format your in-text citations and generate a
list of references; it knows most standard styles, and you can design your own.
\end{itemize}
\end{frame}

%%%%%%%%%%%%%%%%%%%%%%%%%%%%%%%%%%%%%%%%%%%%%%%%%%%%%%%%%%%%%%%%%%%%%%%%%%%%%%%
%%%%%%%%%%%%%%%%%%%%%%%%%%%%%%%%%%%%%%%%%%%%%%%%%%%%%%%%%%%%%%%%%%%%%%%%%%%%%%%
%%%%%%%%%%%%%%%%%%%%%%%%%%%%%%%%%%%%%%%%%%%%%%%%%%%%%%%%%%%%%%%%%%%%%%%%%%%%%%%
\begin{frame}[fragile]{\insertsubsection{} 3 (Optional)}
\begin{itemize}
\item Use the \bftt{natbib} package\footnote{There is a new package with more
  features named \bftt{biblatex} but most of the articles templates still use
  \bftt{natbib}.} with \cmdbs{citet} and \cmdbs{citep}.
\item Reference \cmdbs{bibliography} at the end, and specify a \cmdbs{bibliographystyle}.
\end{itemize}
\begin{minipage}{0.55\linewidth}
\inputminted[fontsize=\scriptsize,frame=single,resetmargins]{latex}%
  {bib-example.tex}
\end{minipage}
\begin{minipage}{0.35\linewidth}
\includegraphics[width=\textwidth,clip,trim=1.8in 5in 1.8in 1in]{bib-example.pdf}
\end{minipage}
\end{frame}

%%%%%%%%%%%%%%%%%%%%%%%%%%%%%%%%%%%%%%%%%%%%%%%%%%%%%%%%%%%%%%%%%%%%%%%%%%%%%%%
%%%%%%%%%%%%%%%%%%%%%%%%%%%%%%%%%%%%%%%%%%%%%%%%%%%%%%%%%%%%%%%%%%%%%%%%%%%%%%%
%%%%%%%%%%%%%%%%%%%%%%%%%%%%%%%%%%%%%%%%%%%%%%%%%%%%%%%%%%%%%%%%%%%%%%%%%%%%%%%
\subsection{Exercise}
\begin{frame}[fragile]{Exercise: Putting it All Together}

Add an image and a bibliography to the paper from the previous exercise.

\begin{enumerate}
\item Download these example files to your computer.

\begin{center}
\fbox{\href{\fileuri/gerbil.jpg?dl=1}{Click to download example image}}

\fbox{\href{\fileuri/bib-exercise.bib?dl=1}{Click to download example bib file}}
\end{center}

\item Upload them to Overleaf (use the project menu).

\end{enumerate}
\end{frame}

%%%%%%%%%%%%%%%%%%%%%%%%%%%%%%%%%%%%%%%%%%%%%%%%%%%%%%%%%%%%%%%%%%%%%%%%%%%%%%%
%%%%%%%%%%%%%%%%%%%%%%%%%%%%%%%%%%%%%%%%%%%%%%%%%%%%%%%%%%%%%%%%%%%%%%%%%%%%%%%
%%%%%%%%%%%%%%%%%%%%%%%%%%%%%%%%%%%%%%%%%%%%%%%%%%%%%%%%%%%%%%%%%%%%%%%%%%%%%%%
\section{What's Next?}

%%%%%%%%%%%%%%%%%%%%%%%%%%%%%%%%%%%%%%%%%%%%%%%%%%%%%%%%%%%%%%%%%%%%%%%%%%%%%%%
%%%%%%%%%%%%%%%%%%%%%%%%%%%%%%%%%%%%%%%%%%%%%%%%%%%%%%%%%%%%%%%%%%%%%%%%%%%%%%%
%%%%%%%%%%%%%%%%%%%%%%%%%%%%%%%%%%%%%%%%%%%%%%%%%%%%%%%%%%%%%%%%%%%%%%%%%%%%%%%
\begin{frame}{Outline}
\begin{multicols}{2}
\tableofcontents[currentsection]
\end{multicols}
\end{frame}

%%%%%%%%%%%%%%%%%%%%%%%%%%%%%%%%%%%%%%%%%%%%%%%%%%%%%%%%%%%%%%%%%%%%%%%%%%%%%%%
%%%%%%%%%%%%%%%%%%%%%%%%%%%%%%%%%%%%%%%%%%%%%%%%%%%%%%%%%%%%%%%%%%%%%%%%%%%%%%%
%%%%%%%%%%%%%%%%%%%%%%%%%%%%%%%%%%%%%%%%%%%%%%%%%%%%%%%%%%%%%%%%%%%%%%%%%%%%%%%
\subsection{More Neat Things (Optional)}
\begin{frame}[fragile]{\insertsubsection}
\begin{itemize}
\item Add the \cmdbs{tableofcontents} command to generate a table of contents
from the \cmdbs{section} commands.

\item Change the \cmdbs{documentclass} to
\mint{latex}!\documentclass{scrartcl}!
or
\mint{latex}!\documentclass[12pt]{IEEEtran}!

\item Define your own command for a complicated equation:
\begin{exampletwouptiny}
\newcommand{\rperf}{%
  \rho_{\text{perf}}}
$$
\rperf = {\bf c}'{\bf X} + \varepsilon
$$
\end{exampletwouptiny}
\end{itemize}
\end{frame}

%%%%%%%%%%%%%%%%%%%%%%%%%%%%%%%%%%%%%%%%%%%%%%%%%%%%%%%%%%%%%%%%%%%%%%%%%%%%%%%
%%%%%%%%%%%%%%%%%%%%%%%%%%%%%%%%%%%%%%%%%%%%%%%%%%%%%%%%%%%%%%%%%%%%%%%%%%%%%%%
%%%%%%%%%%%%%%%%%%%%%%%%%%%%%%%%%%%%%%%%%%%%%%%%%%%%%%%%%%%%%%%%%%%%%%%%%%%%%%%
\subsection{More Neat Packages}
\begin{frame}{\insertsubsection}
\begin{itemize}
\item \bftt{beamer}: for presentations (like this one!)
\item \bftt{todonotes}: comments and TODO management
\item \bftt{tikz}: make amazing graphics
\item \bftt{pgfplots}: create graphs in \LaTeX
\item \bftt{listings}: source code printer for \LaTeX
\item \bftt{spreadtab}: create spreadsheets in \LaTeX
\item \bftt{gchords}, \bftt{guitar}: guitar chords and tabulature
\item \bftt{cwpuzzle}: crossword puzzles
\end{itemize}
See \url{https://www.overleaf.com/latex/examples} and \url{http://texample.net}
for examples of (most of) these packages.
\end{frame}

%%%%%%%%%%%%%%%%%%%%%%%%%%%%%%%%%%%%%%%%%%%%%%%%%%%%%%%%%%%%%%%%%%%%%%%%%%%%%%%
%%%%%%%%%%%%%%%%%%%%%%%%%%%%%%%%%%%%%%%%%%%%%%%%%%%%%%%%%%%%%%%%%%%%%%%%%%%%%%%
%%%%%%%%%%%%%%%%%%%%%%%%%%%%%%%%%%%%%%%%%%%%%%%%%%%%%%%%%%%%%%%%%%%%%%%%%%%%%%%
\subsection{Installing \LaTeX{}}
\begin{frame}{\insertsubsection}
\begin{itemize}
\item To run \LaTeX{} on your own computer, you'll want to use a \LaTeX{}
\emph{distribution}. A distribution includes a \bftt{latex} program
and (typically) several thousand packages.
\begin{itemize}
\item On Windows: \href{http://miktex.org/}{Mik\TeX} or \href{http://tug.org/texlive/}{\TeX Live}
\item On Linux: \href{http://tug.org/texlive/}{\TeX Live}
\item On Mac: \href{http://tug.org/mactex/}{Mac\TeX}
\end{itemize}
\item You'll also want a text editor with \LaTeX{} support. See \url{http://en.wikipedia.org/wiki/Comparison_of_TeX_editors} for a list of (many) options.
\item You'll also have to know more about how \bftt{latex} and its related tools
work --- see the resources on the next slide.
\end{itemize}
\end{frame}

%%%%%%%%%%%%%%%%%%%%%%%%%%%%%%%%%%%%%%%%%%%%%%%%%%%%%%%%%%%%%%%%%%%%%%%%%%%%%%%
%%%%%%%%%%%%%%%%%%%%%%%%%%%%%%%%%%%%%%%%%%%%%%%%%%%%%%%%%%%%%%%%%%%%%%%%%%%%%%%
%%%%%%%%%%%%%%%%%%%%%%%%%%%%%%%%%%%%%%%%%%%%%%%%%%%%%%%%%%%%%%%%%%%%%%%%%%%%%%%
\subsection{Online Resources}
\begin{frame}{\insertsubsection}
\begin{itemize}
\item \href{http://en.wikibooks.org/wiki/LaTeX}{The \LaTeX{} Wikibook} ---
excellent tutorials and reference material.
\item \href{http://tex.stackexchange.com/}{\TeX{} Stack Exchange} --- ask
questions and get excellent answers incredibly quickly
\item \href{http://www.latex-community.org/}{\LaTeX{} Community} --- a large
online forum
\item \href{http://ctan.org/}{Comprehensive \TeX{} Archive Network (CTAN)} ---
over four thousand packages plus documentation
\item Google will usually get you to one of the above.
\end{itemize}
\end{frame}

%%%%%%%%%%%%%%%%%%%%%%%%%%%%%%%%%%%%%%%%%%%%%%%%%%%%%%%%%%%%%%%%%%%%%%%%%%%%%%%
%%%%%%%%%%%%%%%%%%%%%%%%%%%%%%%%%%%%%%%%%%%%%%%%%%%%%%%%%%%%%%%%%%%%%%%%%%%%%%%
%%%%%%%%%%%%%%%%%%%%%%%%%%%%%%%%%%%%%%%%%%%%%%%%%%%%%%%%%%%%%%%%%%%%%%%%%%%%%%%
\begin{frame}
\begin{center}
Thanks, and happy \TeX{}ing!
\end{center}
\end{frame}

\end{document}

% -- latex understands words, sentences and paragraphs

Words are separated by one or more spaces.  Paragraphs are separated by
one or more blank lines.  The output is not affected by adding extra
spaces or extra blank lines to the input file.

Double quotes are typed like this: ``quoted text''.
Single quotes are typed like this: `single-quoted text'.

Emphasized text is typed like this: \emph{this is emphasized}.
Bold       text is typed like this: \textbf{this is bold}.

-- Adding structure to your document

\section{Hello}

\subsection{World}

\subsection{Foo}

\subsubsection*{Stuff} % star form

\subsubsection*{Results}

-- Labels and cross-references

\label{sec:intro}
\label{sec:method}
\ref{sec:method}

--> maybe introduce the prettyref package here.

-- Mathematics

Inline mathematics: $x + y < 7$.

'Displayed' mathematics:
\begin{equation}
\end{equation}

\begin{equation*}
\end{equation*}

\begin{align}
\end{align}

-- Figures

- Need the graphicx package.

- here we can start introducing options

\includegraphics[width=\textwidth]{}

- where do you find out about these options? --> link to the Wikibook

-- Floating Figures

\begin{figure}
\includegraphics{...}
\caption{\label{}Here is a caption.}
\end{figure}

-- Tables

- not the nicest part of LaTeX

\usepackage{tabularx}

\begin{tabular}{llr}
Item & Quantity & Price (\$) & Amount
Widget & 1 &
\end{tabular}

Bonus points: check out the fp package and the spreadtab package.

-- Document Classes

a .cls file

article

some journal templates come with one

-- Bibliographies



-- For Typesetting Geeks

- dashes: -, --, ---

- ellipsis.

- controlling spaces: ~, \ , \,, \@

- spacing after periods (et al., etc.)

- Nested quotation marks: ``\,`
\vskip 2ex
\item Use the \emph{star form} to display an equation without a number.
\begin{exampletwouptiny}
\begin{equation*}
F(x) = \int_{a}^{x}{f(t) dt}
\end{equation*}
\end{exampletwouptiny}

\begin{itemize}
\item \bftt{equation} and \bftt{equation*} are called \emph{environments}.
\begin{itemize}
  \item The \cmdbs{begin} and \cmdbs{end} commands define the environment.
  \item The \cmd{\$} also starts and ends an environment.
  \item Some commands are defined only within certain environments.
  \item Some commands behave differently in different environments.
\end{itemize}
\end{itemize}
\end{block}
\begin{center}
\fbox{\href{http://ctan.org/}{The Comprehensive \TeX Archive Network (CTAN)}}
\end{center}

%%%%%%%%%%%%%%%%%%%%%%%%%%%%%%%%%%%%%%%%%%%%%%%%%%%%%%%%%%%%%%%%%%%%%%%%%%%%%%%
%%%%%%%%%%%%%%%%%%%%%%%%%%%%%%%%%%%%%%%%%%%%%%%%%%%%%%%%%%%%%%%%%%%%%%%%%%%%%%%
%%%%%%%%%%%%%%%%%%%%%%%%%%%%%%%%%%%%%%%%%%%%%%%%%%%%%%%%%%%%%%%%%%%%%%%%%%%%%%%
\subsection{Typography tweaks}
\begin{frame}{\insertsubsection}
\begin{tabular}{lll}
& character name & used mainly for \ldots \\\hline
\bftt{\bs} & backslash                 & commands, tables \\
\bftt{\{}  & open brace                & commands \\
\bftt{\}}  & close brace               & commands \\
\bftt{\%}  & percent sign              & comments \\
\bftt{\#}  & hash (pound / sharp) sign & custom commands \\
\bftt{\$}  & dollar sign               & equations \\
\bftt{\_}  & underscore                & equations (subscripts) \\
\bftt{\^}  & caret                     & equations (superscripts) \\
\bftt{\&}  & ampersand                 & tables \\
\bftt{\~}  & tilde                     & spacing \\
\end{tabular}
\end{frame}

%\item We've used several environments:
%\vskip 1ex
%{\scriptsize
%\begin{tabular}{ll}
%\cmdbs{begin}\bftt{\{document\}}\ldots\cmdbs{end}\bftt{\{document\}} &
%  document environment \\
%\cmdbs{begin}\bftt{\{itemize\}}\ldots\cmdbs{end}\bftt{\{itemize\}} &
%  itemized list environment \\
%\bftt{\$\ldots\$}     & \emph{in-text} math environment \\
%\bftt{\$\$\ldots\$\$} & \emph{displayed} math environment \\
%\cmdbs{begin}\bftt{\{equation\}}\ldots\cmdbs{end}\bftt{\{equation\}} &
%  displayed math environment w/ number
%\end{tabular}
%}
